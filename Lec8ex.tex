\documentclass[10pt,CCT]{ctexart}
\textheight=23cm \textwidth=16cm \oddsidemargin .5cm \topmargin
-2cm




\setlength{\parindent}{2\ccwd}          %段落首行缩进量 \ccwd表示一个汉定的宽度
\setlength{\parskip}{2pt plus1pt minus1pt}  %段落之间的竖直距离
\setlength{\baselineskip}{20pt plus2pt minus1pt}%段落内的行距
\setlength{\textheight}{22true cm}      %每页上的文本总高度
\setlength{\textwidth}{15true cm}     %每页上的文本总宽度

\newcommand{\beq}{\begin{eqnarray}}
\newcommand{\eeq}{\end{eqnarray}}
\newcommand{\be}{\begin{equation}}
\newcommand{\ee}{\end{equation}}
\newcommand{\bq}{\begin{eqnarray*}}
\newcommand{\eq}{\end{eqnarray*}}

\newcommand{\D}{\scriptscriptstyle}
\newtheorem{lemma}{{\heiti 引理}}
\newtheorem{theorem}{{\heiti 定理}}
\newtheorem {definition}{{\heiti 定义}}
\newtheorem{corollary}{{\heiti 推论}}
\newtheorem{prob}{~}

\renewcommand{\theequation}{\thesection.\arabic{equation}}
\renewcommand{\thetheorem}{\thesection.\arabic{theorem}}
\renewcommand{\thecorollary}{\thesection.\arabic{corollary}}
\renewcommand{\thedefinition}{\thesection.\arabic{definition}}
\renewcommand{\thelemma}{\thesection.\arabic{lemma}}

% 生成标题,即使上面几条标题命令的定义生效

\begin{document}%

\title{Lec8: 作业题目}
 \date{\today}%

\maketitle
\thispagestyle{empty}
\begin{prob}\rm
使用Metropolis-Hastings算法从Cauchy分布中抽样, 在丢弃开始的1000个burnin样本后, 比较其
百分位数与理论的百分位数(qcauchy或者qt).
\end{prob}

\begin{prob}\rm
考虑\emph{pumps}数据(10个电厂泵, $t_i$: 泵的运行时间(单位, 1000小时), $y_i$: 失效次数), 
使用MCMC算法对下述三种模型, 用后验均值估计$\alpha,\beta$:

$y_i\sim Poisson(\lambda_i)$, 以及

模型1: $log(\lambda_i)=\alpha+\beta t_i$.

模型2: $log(\lambda_i)=\alpha+\beta log(t_i)$.

模型3: $\lambda_i=\theta_it_i, \theta_i\sim Gamma(\alpha,\beta)$.

\end{prob}




\end{document}







